% To be compiled with LaTeX
% Set ``showsol'' to true to show the solutions.

\documentclass[12pt]{article}
\usepackage{amsmath}
\parskip 10pt
%\usepackage{amsfonts}
%\usepackage{latexsym}
%\usepackage{graphpap}
\usepackage{epsfig}

\usepackage{ifthen}

\newboolean{showsol}

%\setboolean{showsol}{false}
\setboolean{showsol}{true}

\newcommand{\ts}{Midterm Exam No.\,1} % Which test is this one?


\newcommand{\solution}[2]{\ifthenelse{\boolean{showsol}}%
{\vskip10pt\noindent\emph{Solution:}\vskip10pt #2}{\vskip #1}}

\newcommand{\pnum}[1]{\noindent {\hskip -18pt \bfseries #1.}}

\makeatletter
\def\ps@myhead{%
    \let\@oddfoot\@empty\let\@evenfoot\@empty
    \def\@evenhead{\hfil{\large\leftmark}\hfil}%
    \def\@oddhead{\hfil{\large\rightmark}\hfil}%
    \let\@mkboth\@gobbletwo
    \let\sectionmark\@gobble
    \let\subsectionmark\@gobble
    \def\@oddfoot{\reset@font\hfil\thepage\hfil}
    \let\@evenfoot\@oddfoot
    }
\makeatother


\begin{document}
%\setlength{\unitlength}{1mm}
\thispagestyle{myhead}

\ifthenelse{\boolean{showsol}}
{
\markright{\bfseries Math B17 
- Winter 1999 - \ts{} (solutions)}
}
{
\markright{\bfseries B17 
- Winter 1999 - \ts{} }
}



%begin header
\ifthenelse{\boolean{showsol}}
{
\centerline{\Large SOLUTIONS}
}
{
\noindent \begin{minipage}[t]{3.5 in}
\Large
\begin{tabular}{|r|c|}\hline 
{\bfseries  Last Name:}&{ \ \ \ \ \ \ \ \ \ \ \ \ \ \ \ \ \ \ \ \ \ \ \ \ \ \ \ \ \ \ \ \ \ \ \ \ \ \ \ \ \ \ \ \ \ \ \ \ \ \ }\\ 
\hline 
{\bfseries First Name:}&{ \ \ \ \ \ \ \ \ \ \ \ \ \ \ \ \ \ \ \ \ \ \ \ \ \ \ \ \ \ \ \ \ \ \ \ \ \ \ \ \ \ \ \ \ \ \ \ \ \ \ }\\ 
\hline
\end{tabular}
\end{minipage}

\bigskip

%{\noindent \bfseries Show all your work. 
%
%\noindent No calculators allowed.}
}%end header
%
\bigskip


%%%%%%%%%%%%%%%%%%%%%%%%%%%%%%%%%%%%%%%%%%%%%%%%%%%%%%%%%%%%%%%%


\pnum{1} Determine if the following infinite series converges 
or diverges:

$\displaystyle{\sum_{n=1}^\infty\,\frac{\ln{n}}{1+\ln{(n+7)}}}$

\solution{2.5in}{\par
Using l'H\^opital's rule we check that the $n$-th term does not 
converge to zero:
\[
\lim_{n\to \infty}\,\frac{\ln{n}}{1+\ln{(n+7)}} =
\lim_{x\to \infty}\,\frac{\ln{x}}{1+\ln{(x+7)}} =
\lim_{x\to \infty}\,\frac {1/x}{1/(x+7)} = 
\lim_{x\to \infty}\,\frac {x+7}{x} = 1
\]

Hence, by the $n$-th Term Test for Divergence, the series diverges.
}

\clearpage

%%%%%%%%%%%%%%%%%%%%%%%%%%%%%%%%%%%%%%%%%%%%%%%%%%%%%%%%%%%%%%%%


\pnum{2} Use the integral test to determine if the 
following series converges or diverges:

$\displaystyle{\sum_{n=2}^\infty\,\frac{1}{n\,(\ln{n})^2}}$

\solution{2.5in}{\par First, note that 
$f(x) = \frac{1}{x\,(\ln{x})^2}$ 
is continuous, positive and decreasing for
$x\geq 2$. Next, we compute the following integral:
\[
\int_2^n\,\frac{1}{x\,(\ln{x})^2}\,dx = 
\left[-\frac{1}{\ln{x}}\right]_{2}^n =
\frac{1}{\ln{2}} - \frac{1}{\ln{n}}\ .
\]
So:
\[
\lim_{n\to \infty}\,\int_2^n\,\frac{1}{x\,(\ln{x})^2} = 
\frac{1}{\ln{2}}\ .
\]

Since the integral converges, the series converges.
}


\clearpage


%%%%%%%%%%%%%%%%%%%%%%%%%%%%%%%%%%%%%%%%%%%%%%%%%%%%%%%%%%%%%%%%

\pnum{3} Let $S$ be the sum of the following series:

$\displaystyle{S = \sum_{n=0}^\infty\,\frac{\cos^2{n}}{5^n}}$

Determine which one of the following statements is true and 
show why:

\begin{enumerate}
\item The series diverges.
\item The series converges and $5/4 \leq S$.
\item \label{L} The series converges and $0 < S < 5/4$.
\end{enumerate}

\solution{2.5in}{\par
First note that $0 \leq \cos^2{n} \leq 1$, hence:
\[
0 \leq \frac{\cos^2{n}}{5^n} \leq \frac{1}{5^n}
\]

Since the following geometric series converges:
\[
\sum_{n=0}^\infty\,\frac{1}{5^n} = 
\frac{1}{1-\frac{1}{5}} = \frac{5}{4}\ ,
\]
by Comparison Test the given series also converges, and its sum is 
$S\leq 5/4$.  Note that the inequality is actually strict 
($S < 5/4$), since, for instance, 
$\frac{\cos^2{1}}{5} < 1/5$.  
Hence statement \ref{L} is true.  }


\clearpage

%%%%%%%%%%%%%%%%%%%%%%%%%%%%%%%%%%%%%%%%%%%%%%%%%%%%%%%%%%%%%%%%


\pnum{4} Find the interval of convergence of the following 
power series:

$\displaystyle{\sum_{n=1}^\infty\,\frac{(-1)^n}{n}\,(x-2)^n}$

\solution{2.5in}{\par

By the method at the beginning of section 11.8 of the textbook:
\[
\rho = 
\lim_{n\to\infty}\,\frac{1/(n+1)}{1/n} =
\lim_{n\to\infty}\,\frac{n}{n+1} = 1\ ,
\]
hence the radius of convergence is $R = 1/\rho = 1$, so the power
series converges absolutely for $|x-2|<1$, i.e., $1 < x < 3$.

Alternatively, using directly the Ratio Test, the series converges
absolutely wherever the following limit is less than 1:
\[
\lim_{n\to\infty}\,\frac{|x-2|^{n+1}/(n+1)}{|x-2|^n/n} =
\lim_{n\to\infty}\,\frac{n}{n+1}\,|x-2| = |x-2|\ ,
\]
hence the power series converges absolutely for $|x-2|<1$, i.e., 
$1 < x < 3$.

Next we test the endpoints. 

For $x=1$ the series is
\[
\sum_{n=1}^\infty\,\frac{1}{n}\ ,
\]
which diverges.

For $x=3$ the series becomes 
\[
\sum_{n=1}^\infty\,\frac{(-1)^n}{n}\ ,
\]
which converges.

Hence its interval of convergence is $(1,3]$.
}


\clearpage


%%%%%%%%%%%%%%%%%%%%%%%%%%%%%%%%%%%%%%%%%%%%%%%%%%%%%%%%%%%%%%%%


\pnum{5} Find the power series in $x$ of the function defined 
by the following integral:

$\displaystyle{f(x) = \int_0^x \frac{\sin{t}}{t}}\,dt =$

\solution{2.5in}{\par
The power series of $\sin t$ is:
\[
\sin{t} = \sum_{n=0}^\infty\,(-1)^n\,\frac{t^{2n+1}}{(2n+1)!}
= t - \frac{t^3}{3!} + \frac{t^5}{5!} - \frac{t^7}{7!} + \cdots
\]

Dividing by $t$ we get:
\[
\frac{\sin{t}}{t} = \sum_{n=0}^\infty\,(-1)^n\,\frac{t^{2n}}{(2n+1)!}
= 1 - \frac{t^2}{3!} + \frac{t^4}{5!} - \frac{t^6}{7!} + \cdots
\]

Integrating termwise we get:
\[
\int_0^x\,\frac{\sin{t}}{t}\,dt = 
\sum_{n=0}^\infty\,(-1)^n\,\frac{x^{2n+1}}{(2n+1)!\,(2n+1)} = 
x - \frac{x^3}{3!\,3} + \frac{x^5}{5!\,5} - \frac{x^7}{7!\,7} + \cdots
\]
}


\clearpage


%%%%%%%%%%%%%%%%%%%%%%%%%%%%%%%%%%%%%%%%%%%%%%%%%%%%%%%%%%%%%%%%


\pnum{6} Use power series to compute the following limit:

$\displaystyle{\lim_{x\to 0}\,
     \left(\,\frac{1}{x} - \frac{1}{\sin{x}}\,\right) =}$

\solution{2.5in}{\par
\[
\begin{aligned}
\lim_{x\to 0}\,
     \left(\,\frac{1}{x} - \frac{1}{\sin{x}}\,\right) 
&=
\lim_{x\to 0}\,
     \frac{\sin{x} - x}{x\,\sin{x}} \\
&=
\lim_{x\to 0}\,
     \frac{(x-\frac{x^3}{3!} + \frac{x^5}{5!} -\cdots) - x}
                 {x\,(x-\frac{x^3}{3!} + \frac{x^5}{5!} -\cdots)} \\
&= 
\lim_{x\to 0}\,
     \frac{-\frac{x^3}{3!} + \frac{x^5}{5!} -\cdots}
                 {x^2-\frac{x^4}{3!} + \frac{x^6}{5!} -\cdots} \\
&=
\lim_{x\to 0}\,
     \frac{-\frac{x}{3!} + \frac{x^3}{5!} -\cdots}
                 {1-\frac{x^2}{3!} + \frac{x^4}{5!} -\cdots} \\
&= \frac{0}{1} = 0
\end{aligned}
\]
}


\clearpage


%%%%%%%%%%%%%%%%%%%%%%%%%%%%%%%%%%%%%%%%%%%%%%%%%%%%%%%%%%%%%%%%




\end{document}






