
\documentclass{article}
\parskip 8pt
%\usepackage{amsfonts}
%\usepackage{latexsym}
%\usepackage{graphpap}
%\usepackage{epsfig}

\begin{document}
\setlength{\unitlength}{1mm}
\pagestyle{myheadings}
\markright{\bfseries M305G \#50470 - 
Spring 1997 - Lerma - Practice Test 1 (answers)}
\bigskip

\centerline{SOLUTIONS}

\bigskip


\bigskip
\noindent 1. Find $\displaystyle \frac {f(x+h)-f(x)} {h}$ 
for the following functions:

\noindent (a) $f(x) = 5-x^2$

$\displaystyle 
\frac {5-(x+h)^2-(5-x^2)} {h} = \frac {5-x^2-2xh-h^2-5+x^2} {h}
= -2x-h
$

\bigskip

\noindent (b) $f(x) = 5x+3$

$\displaystyle 
\frac {5(x+h)+3-(5x+3)} {h} = \frac {5x+5h+3-5x-3} {h} = 5
$

\bigskip

\medskip

\noindent 2. Find the domain and range of 
$\displaystyle y = \sqrt{\frac {x+5}{1-x}}$

\noindent (a) DOMAIN: We need $1-x\neq 0 \rightarrow x\neq 0$ and 
$\frac {x+5} {1-x} \geq 0$. Study of signs: 

\begin{center}
\begin{tabular}{|l|c|c|c|}
\hline
&$(-\infty,-5)$ & $(-5,1)$ & $(1,\infty)$\\
\hline
$x+5$ & $-$ & $+$ & $+$\\
\hline
$1-x$ & $+$ & $+$ & $-$\\
\hline
$\frac {x+5} {1-x}$ & $-$ & $+$ & $-$\\
\hline
\end{tabular}
\end{center}

Hence $D = [-5,0)$

\noindent (b) RANGE: $\displaystyle y = \sqrt{\frac {x+5}{1-x}}$; 
$\displaystyle y^2 = \frac {x+5}{1-x} \quad (y\geq 0)$;
$\displaystyle y^2 (1-x) = x+5 \quad (y\geq 0)$;
$\displaystyle y^2 - xy^2 = x+5 \quad (y\geq 0)$;
$\displaystyle y^2 - 5 = x (1+y^2) \quad (y\geq 0)$; 
$\displaystyle x = \frac {y^2-5} {1+y^2} \quad (y\geq 0)$. 
We need $\displaystyle 1+y^2 \geq 0$, but this is always true, 
so there are no additional restritions to $y\geq 0$. 

Hence $R = [0,\infty)$

\medskip

\noindent 3. Find the x- and y- intercepts for the following curves:

\noindent (a) $\displaystyle y = (x-2)(3x+5)(x+2)$

\noindent x-intercepts: $y=0$; $(x-2)(3x+5)(x+2)=0$; 
$x = 2, -\frac 5 3, -2$. Hence:
$$
(2,0)\quad(-\frac 5 3,0)\quad(-2,0)
$$

\noindent y-intercepts: $x=0$; $y = (-2)(5)(2) = -20$. Hence:
$$
(0,-20)
$$

\noindent (b) $\displaystyle \frac {(x-1)^2}{4} + \frac {(y+1)^2}{9} = 1$

x-intercepts: $y=0$; $\displaystyle \frac {(x-1)^2}{4} + \frac 1 9 = 1$;
$x=1\pm \frac 4 3 \sqrt{2}$.
$$
(1 + \frac 4 3 \sqrt{2},0)\quad(1 - \frac 4 3 \sqrt{2},0)
$$

y-intercepts: $x=0$; $\frac 1 4 + \frac {(y+1)^2}{9} = 1$;
$y = -1 \pm \frac 3 2 \sqrt{3}$.
$$
(0,-1 + \frac 3 2 \sqrt{3})\quad(0,-1 - \frac 3 2 \sqrt{3})
$$

\medskip

\noindent 4. Given $\displaystyle f(x) = 9 x^2$ and 
$\displaystyle (h,k)=(-\sqrt{2},3)$, write $\displaystyle y-k=f(x-h)$.

$$
y - 3 = 9 (x + \sqrt{2})^2
$$

\medskip

\noindent 5. Fin de sum $f+g$, difference $f-g$, product $fg$ 
and quotient $f/g$ of the functions 
$\displaystyle f(x) = x^2$ and $\displaystyle g(x) = 5x-2$.

  $\displaystyle (f+g)(x) = x^2 + 5x - 2$

  $\displaystyle (f-g)(x) = x^2 - 5x + 2$

  $\displaystyle (fg)(x) = 5x^3 - 2x^2$

  $\displaystyle (f/g)(x) = \frac {x^2}{5x-2}$

\medskip

\noindent 6. Given the functions $\displaystyle f(x) = 3x-1$ and 
$\displaystyle g(x) = 5-x^2$, find the compositions:

\noindent (a) $\displaystyle g \circ f$

  $\displaystyle (g \circ f)(x) = -9x^2+6x+4$

\noindent (b) $\displaystyle f \circ g$

  $\displaystyle (f \circ g)(x) = 14 - 3x^2$

\medskip

\noindent 7. Compute by sythetic division: 
$\displaystyle \frac {3x^3+2x^2-12x-8} {x-2}$

\begin{center}
\begin{tabular}{rrrrr}
$2:$ & $3$ & $\ 2$ & $-12$ & $-8$\\
 & & $6$ & $16$ & $8$\\
\hline
 & $3$ & $8$ & $4$ & $0$
\end{tabular}
\end{center}

REMAINDER = 0.

QUOTIENT = $3x^2 +8x + 4$.

\medskip

\noindent 8. Use slopes to decide if the triangle with vertices 
$\displaystyle A(12,-7)$, $\displaystyle B(4,3)$, 
$\displaystyle C(-3,-2)$ is a right triangle.

$\displaystyle AB = (-8,10)$; $m_1 = -\frac 5 4$.

$\displaystyle BC = (-7,-5)$; $m_2 = \frac 5 7$.

$\displaystyle CA = (15,-5)$; $m_3 = -\frac {-1} 3$.

Note that none of $m_1 m_2$, $m_2 m_3$, $m_3 m_1$, is $-1$, 
so the triangle is not a right triangle.

\medskip

\noindent 9. Find the maximum or minimun value for the following 
quadratic function, and state if it is a maximum or a minimum: 
$\displaystyle 2 x^2 +24 x + y = 178$

\noindent $\displaystyle y = -2x^2 - 24x + 178$; 
$\displaystyle y - 250 = -2(x+6)$. 

\noindent The vertex is at $(-6,250)$, and it is a maximum because 
the leading coefficient is negative, which implies that 
the parabole is open down.

\medskip

\noindent 10. Graph the function $\displaystyle f(x) = x^3 - 6 x^2$

(a) x-intercepts: $x^2(x-6)=0$; $x=0$ or $x=6$.

(b) y-intercepts: $x=0$; $y=0$.

(c) For large $x$ the function behaves as $x^3$.

(d) Since the degree is 3, there are at most 2 turning points.

(Graph not included).


\end{document}






