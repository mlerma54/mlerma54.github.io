% To be compiled with LaTeX
% Set ``showsol'' to true to show the solutions.

\documentclass[12pt]{article}
\usepackage{amsmath}
\parskip 10pt
\usepackage{amsfonts}
%\usepackage{latexsym}
%\usepackage{graphpap}
\usepackage{epsfig}

\usepackage{ifthen}

\newboolean{showsol}

%\setboolean{showsol}{false}
\setboolean{showsol}{true}

\newcommand{\ts}{Midterm Exam No.\,2} % Which test is this one?


\newcommand{\solution}[2]{\ifthenelse{\boolean{showsol}}%
{\vskip10pt\noindent\emph{Solution:}\vskip10pt #2}{\vskip #1}}

\newcommand{\pnum}[1]{\noindent {\hskip -18pt \bfseries #1.}}

\makeatletter
\def\ps@myhead{%
    \let\@oddfoot\@empty\let\@evenfoot\@empty
    \def\@evenhead{\hfil{\large\leftmark}\hfil}%
    \def\@oddhead{\hfil{\large\rightmark}\hfil}%
    \let\@mkboth\@gobbletwo
    \let\sectionmark\@gobble
    \let\subsectionmark\@gobble
    \def\@oddfoot{\reset@font\hfil\thepage\hfil}
    \let\@evenfoot\@oddfoot
    }
\makeatother


\begin{document}
%\setlength{\unitlength}{1mm}
\thispagestyle{myhead}

\ifthenelse{\boolean{showsol}}
{
\markright{\bfseries Math B17 
- Winter 1999 - \ts{} (solutions)}
}
{
\markright{\bfseries B17 
- Winter 1999 - \ts{} }
}



%begin header
\ifthenelse{\boolean{showsol}}
{
\centerline{\Large SOLUTIONS}
}
{
\noindent \begin{minipage}[t]{3.5 in}
\Large
\begin{tabular}{|r|c|}\hline 
{\bfseries  Last Name:}&{ \ \ \ \ \ \ \ \ \ \ \ \ \ \ \ \ \ \ \ \ \ \ \ \ \ \ \ \ \ \ \ \ \ \ \ \ \ \ \ \ \ \ \ \ \ \ \ \ \ \ }\\ 
\hline 
{\bfseries First Name:}&{ \ \ \ \ \ \ \ \ \ \ \ \ \ \ \ \ \ \ \ \ \ \ \ \ \ \ \ \ \ \ \ \ \ \ \ \ \ \ \ \ \ \ \ \ \ \ \ \ \ \ }\\ 
\hline
\end{tabular}
\end{minipage}

\bigskip

%{\noindent \bfseries Show all your work. 
%
%\noindent No calculators allowed.}
}%end header
%
\bigskip


%%%%%%%%%%%%%%%%%%%%%%%%%%%%%%%%%%%%%%%%%%%%%%%%%%%%%%%%%%%%%%%%

\pnum{1} Let $A$, $B$ and $C$ be the following matrices:
\[
A =
\left [\begin {array}{rr} 1&4\\\noalign{\medskip}3&6\end {array}
\right ]
\qquad
B =
\left [\begin {array}{rr} 1&-3\\\noalign{\medskip}1&5\end {array}
\right ]
\qquad
C =
\left [\begin {array}{rr} -2&5\\\noalign{\medskip}1&-2\end {array}
\right ]
\]
Compute $(3A+B)\,C$.


\solution{2.5in}{\par
\[
\begin{aligned}
(3A + B ) \, C &=
\left(
\left [\begin {array}{rr} 3&12\\\noalign{\medskip}9&18\end {array}
\right ]
+
\left [\begin {array}{rr} 1&-3\\\noalign{\medskip}1&5\end {array}
\right ]
\right)
\left [\begin {array}{rr} -2&5\\\noalign{\medskip}1&-2\end {array}
\right ]
\\ \noalign{\medskip}
&=
\left [\begin {array}{rr} 4&9\\\noalign{\medskip}10&23\end {array}
\right ]
\left [\begin {array}{rr} -2&5\\\noalign{\medskip}1&-2\end {array}
\right ]
\\ \noalign{\medskip}
&=
\left [\begin {array}{rr} 1&2\\\noalign{\medskip}3&4\end {array}
\right ]
\end{aligned}
\]

}

\clearpage

%%%%%%%%%%%%%%%%%%%%%%%%%%%%%%%%%%%%%%%%%%%%%%%%%%%%%%%%%%%%%%%%


\pnum{2} 
Is the vector 
$\displaystyle{
\mathbf{b} = 
\left [\begin {array}{r} 2\\\noalign{\medskip}1\end {array}\right ]
}$
in the column space of the matrix
$\displaystyle{
A =
\left [\begin {array}{ccc} 1&3&5\\\noalign{\medskip}1&3&2\end {array}
\right ] ?
}$

\solution{2.5in}{\par 
The vector $\mathbf{b}$ is in the column space of $A$ iff the 
system $A\,\mathbf{v} = \mathbf{b}$ has a solution. The augmented 
matrix is:
\[
A' = 
\left [\begin {array}{rrr} 1&3&5\\\noalign{\medskip}1&3&2
\end {array}\right.
\hskip -2pt
\left|
\begin {array}{r} 
2\\\noalign{\medskip}1
\end {array}\right]
\]

After using Gauss-Jordan reduction it becomes:
$\displaystyle{
\left [\begin {array}{rrr} 1&3&0\\\noalign{\medskip}0&0&1
\end {array}\right .
\hskip -2pt
\left|
\begin {array}{r} 
\frac{1}{3}\\\noalign{\medskip}\frac{1}{3}
\end {array}\right]
}$

Here we see that $\mathrm{rank}\,{A} = \mathrm{rank}\,{A'} = 2$, hence
the system has solution and the vector $\mathbf{b}$ does belong to the
column space of the matrix $A$.


}

\clearpage



%%%%%%%%%%%%%%%%%%%%%%%%%%%%%%%%%%%%%%%%%%%%%%%%%%%%%%%%%%%%%%%%



\pnum{3} Solve the following system of equations:

\[
\left\{
\begin{array}{rrrrrrr}
x_{{1}}&+&2\,x_{{2}}&+&3\,x_{{3}}&=& 2 \\
\medskip 
x_{{1}}&+&x_{{2}}&+&x_{{3}}&=& 1 \\
\medskip 
&&x_{{2}}&+&2\,x_{{3}}&=& 1
\end{array}
\right.
\]



\solution{2.5in}{\par
The augmented matrix is:
$\displaystyle{
\left [
\begin{array}{rrr} 1&2&3\\\noalign{\medskip}1&1&1
\\\noalign{\medskip}0&1&2
\end{array}\right.
\hskip -2pt
\left|
\begin{array}{r}
2\\\noalign{\medskip}1\\\noalign{\medskip}1
\end{array}
\right]
}$

After using Gauss-Jordan reduction we get:
$\displaystyle{
\left [
\begin{array}{rrrr} 
1&\phantom{-}0&-1\\
\noalign{\medskip}0&1&2\\
\noalign{\medskip}0&0&0
\end{array}\right.
\hskip -2pt
\left|
\hskip 2pt
\begin{array}{r}
0\\\noalign{\medskip}1\\\noalign{\medskip}0
\end{array}
\right]
}$

i.e.:
\[
\left \{\begin {array}{rrrrrrr} x_1&&&-&x_3&=&0\\
\noalign{\medskip}&&x_2&+&2\,x_3&=&1
\end {array}\right .
\]

The solution is $x_1 = x_3$, $x_2 = 1-2\,x_3$, or:
\[
\left [\begin {array}{r} x_{{1}}\\\noalign{\medskip}x_{{2}}
\\\noalign{\medskip}x_{{3}}\end {array}\right ]
=
\left [\begin {array}{c} 
x_{{3}}\\\noalign{\medskip}1-2\,x_{{3}}\\\noalign{\medskip}x_{{3}}
\end {array}\right ]
=
\left [\begin {array}{r} 0\\\noalign{\medskip}1\\\noalign{\medskip}0
\end {array}\right ]+x_{{3}}\,
\left [\begin {array}{r} 1\\\noalign{\medskip}-2\\\noalign{\medskip}1
\end {array}\right ]
\]

}

\clearpage

%%%%%%%%%%%%%%%%%%%%%%%%%%%%%%%%%%%%%%%%%%%%%%%%%%%%%%%%%%%%%%%%

\pnum{4} Find a basis and the dimension of the solution space 
for the system:
\[
\left\{
\begin {array}{rrrrrrrrrrr} 
x_{{1}}&&&-&3\,x_{{3}}&+&x_{{4}}&-&x_{{5}}&=&0 \\
\noalign{\medskip}&&x_{{2}}&-&x_{{3}}&+&3\,x_{{4}}&+&x_{{5}}&=&0
\end {array}
\right.
\]

\solution{2.5in}{\par
The coefficient matrix is:
\[
A =
\left [\begin{array}{rrrrr} 
1&\phantom{-}0&-3&\phantom{-}1&-1\\\noalign{\medskip}
0&1&-1&3&1
\end{array}\right ]
\]

Note that $A$ is already in Gauss-Jordan reduced form.
Hence, the general solution of $A\mathbf{x} =\mathbf{0}$ is:
\[
\begin{array}{rrrrrrr}
x_1 & = & 3 x_3 & - & x_4 & + & x_5 \\
x_2 & = & x_3 & - & 3 x_4 & - & x_5
\end{array}
\] 
and in matrix form:
$\displaystyle{
\left [\begin {array}{r} x_{{1}}\\\noalign{\medskip}x_{{2}}
\\\noalign{\medskip}x_{{3}}\\\noalign{\medskip}x_{{4}}
\\\noalign{\medskip}x_{{5}}\end {array}\right ]=\ x_{{3}} \left [
\begin {array}{r} 3\\\noalign{\medskip}1\\\noalign{\medskip}1
\\\noalign{\medskip}0\\\noalign{\medskip}0\end {array}\right ]+\ x_{{4}}
\left [\begin {array}{r} -1\\\noalign{\medskip}-3\\\noalign{\medskip}0
\\\noalign{\medskip}1\\\noalign{\medskip}0\end {array}\right ]+\ x_{{5}}
\left [\begin {array}{r} 1\\\noalign{\medskip}-1\\\noalign{\medskip}0
\\\noalign{\medskip}0\\\noalign{\medskip}1\end {array}\right ]\,.
}$

Hence, the following set is a basis of the solution space:
\[
\left\{\quad
\mathbf{v_1} = 
\left [
\begin {array}{r} 3\\\noalign{\medskip}1\\\noalign{\medskip}1
\\\noalign{\medskip}0\\\noalign{\medskip}0\end {array}\right ];
\quad
\mathbf{v_2} = 
\left [\begin {array}{r} -1\\\noalign{\medskip}-3\\\noalign{\medskip}0
\\\noalign{\medskip}1\\\noalign{\medskip}0\end {array}\right ];
\quad
\mathbf{v_3} = 
\left [\begin {array}{r} 1\\\noalign{\medskip}-1\\\noalign{\medskip}0
\\\noalign{\medskip}0\\\noalign{\medskip}1\end {array}\right ]
\quad\right\}
\]
and its dimension is 3.

}

\clearpage

%%%%%%%%%%%%%%%%%%%%%%%%%%%%%%%%%%%%%%%%%%%%%%%%%%%%%%%%%%%%%%%%

\pnum{5} 
Find the inverse of the following matrix:
$\displaystyle{
A = 
\left [\begin {array}{rrr} 1&2&2\\\noalign{\medskip}2&2&1
\\\noalign{\medskip}1&1&1\end {array}\right ] .
}$


\solution{2.5in}{\par
\[
A^{-1} = 
\left [\begin {array}{rrr} -1&0&2\\\noalign{\medskip}1&1&-3
\\\noalign{\medskip}0&-1&2\end {array}\right ]
\]
}

\clearpage


%%%%%%%%%%%%%%%%%%%%%%%%%%%%%%%%%%%%%%%%%%%%%%%%%%%%%%%%%%%%%%%%



\pnum{6} 
Find the eigenvalues and eigenvectors of the following matrix:
$\displaystyle{
A =
\left [\begin {array}{rrr} 3&4&4\\\noalign{\medskip}2&1&-2
\\\noalign{\medskip}-4&-4&-1\end {array}\right ] .
}$


\solution{2.5in}{\par
The characteristic polynomial of $A$ is
\[
\begin{aligned}
\det\,(A-\lambda\,I) = 
\det\left [\begin {array}{ccc} 3-\lambda&4&4\\
\noalign{\medskip}2&1-\lambda&-2
\\\noalign{\medskip}-4&-4&-1-\lambda\end {array}\right ] 
&= -3+3\,{\lambda}^{2}+\lambda-{\lambda}^{3} \\
&=
-\left (\lambda-1\right )\left (\lambda-3\right )\left (\lambda+1
\right )
\end{aligned}
\]

Its roots are $\lambda=3$, $\lambda=1$ and $\lambda=-1$.

For $\lambda=3$ we get
$\displaystyle{
A - 3\,I = 
\left [\begin {array}{rrr} 0&4&4\\\noalign{\medskip}2&-2&-2
\\\noalign{\medskip}-4&-4&-4\end {array}\right ]
}$

After using Gauss-Jordan the matrix becomes:
$\displaystyle{
\left [\begin {array}{ccc} 1&0&0\\\noalign{\medskip}0&1&1
\\\noalign{\medskip}0&0&0\end {array}\right ]
}$

The solutions of $(A-3\,I)\,\mathbf{v} = 0$ 
are:
\[
\mathbf{v} = 
\left [\begin {array}{c} 0\\\noalign{\medskip}-x_{{3}}
\\\noalign{\medskip}x_{{3}}\end {array}\right ]
=
x_3 \left [\begin {array}{r} 0\\\noalign{\medskip}-1\\\noalign{\medskip}1
\end {array}\right ]
\]

So we can take the following eigenvector:
$\displaystyle{
\mathbf{v_1} = 
\left [\begin {array}{r} 0\\\noalign{\medskip}-1\\\noalign{\medskip}1
\end {array}\right ]
}$



For $\lambda=1$ we get
$\displaystyle{
A - I = 
\left [\begin {array}{rrr} 2&4&4\\\noalign{\medskip}2&0&-2
\\\noalign{\medskip}-4&-4&-2\end {array}\right ]
}$

After using Gauss-Jordan the matrix becomes:
$\displaystyle{
\left [\begin {array}{rrr} 1&0&-1\\\noalign{\medskip}0&1&\frac{3}{2}
\\\noalign{\medskip}0&0&0\end {array}\right ]
}$

The solutions of $(A-I)\,\mathbf{v} = 0$ 
are:
\[
\mathbf{v} = 
\left [\begin {array}{c} x_{{3}}\\
\noalign{\medskip}-\frac{3}{2}\,x_{{3}}
\\\noalign{\medskip}x_{{3}}\end {array}\right ]
=
x_3
\left [\begin {array}{r} 1\\\noalign{\medskip}-\frac{3}{2}
\\\noalign{\medskip}1\end {array}\right ]
\]

So we can take the following eigenvector:
$\displaystyle{
\mathbf{v_2} = 
\left [\begin {array}{r} 1\\\noalign{\medskip}-\frac{3}{2}
\\\noalign{\medskip}1\end {array}\right ]
}$


For $\lambda=-1$ we get
$\displaystyle{
A + I = 
\left [\begin {array}{rrr} 4&4&4\\\noalign{\medskip}2&2&-2
\\\noalign{\medskip}-4&-4&0\end {array}\right ]
}$

After using Gauss-Jordan the matrix becomes:
$\displaystyle{
\left [\begin {array}{ccc} 1&1&0\\\noalign{\medskip}0&0&1
\\\noalign{\medskip}0&0&0\end {array}\right ]
}$

The solutions of $(A+I)\,\mathbf{v} = 0$ 
are:
\[
\mathbf{v} = 
\left [\begin {array}{c} -x_{{2}}\\\noalign{\medskip}x_{{2}}
\\\noalign{\medskip}0\end {array}\right ]
=
x_2 \left [\begin {array}{r} -1\\\noalign{\medskip}1
\\\noalign{\medskip}0\end {array}\right ]
\]

So we can take the following eigenvector:
$\displaystyle{
\mathbf{v_3} = 
\left [\begin {array}{r} -1\\\noalign{\medskip}1
\\\noalign{\medskip}0\end {array}\right ]
}$


As a summary, we get the following set of eigenvalues with 
their associated eigenvectors:
\[
\lambda_1 = 3,\quad
\mathbf{v_1} = 
\left [\begin {array}{r} 0\\\noalign{\medskip}-1\\\noalign{\medskip}1
\end {array}\right ]
;
\quad
\lambda_2 = 1,\quad
\mathbf{v_2} = 
\left [\begin {array}{r} 1\\\noalign{\medskip}-\frac{3}{2}
\\\noalign{\medskip}1\end {array}\right ]
;
\quad
\lambda_3 = -1,\quad
\mathbf{v_3} = 
\left [\begin {array}{r} -1\\\noalign{\medskip}1
\\\noalign{\medskip}0\end {array}\right ]
\]


%\[
%\left \{[-1,1,\left \{[-1,1,0]\right \}],[1,1,\left \{[1,-3/2,1]
%\right \}],[3,1,\left \{[0,1,-1]\right \}]\right \}
%\]

}
\clearpage


%%%%%%%%%%%%%%%%%%%%%%%%%%%%%%%%%%%%%%%%%%%%%%%%%%%%%%%%%%%%%%%%




\end{document}






