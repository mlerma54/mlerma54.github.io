% To be compiled with LaTeX
% Set ``showsol'' to true to show the solutions.

\documentclass[12pt]{article}
\usepackage{amsmath}
\parskip 10pt
\usepackage{amsfonts}

\usepackage{ifthen}

\newboolean{showsol}

%\setboolean{showsol}{false}
\setboolean{showsol}{true}

\newcommand{\ts}{Final Exam } % Which test is this one?

\textheight=8.5in

\newcommand{\solution}[2]{\ifthenelse{\boolean{showsol}}%
{\vskip10pt\noindent\emph{Solution:}\vskip10pt #2}{\vskip #1}}

\newlength\answerskip
\answerskip=0in

\newbox\answerbox

\newenvironment{answer}[1]
 {\answerskip=#1 \setbox\answerbox=\vbox\bgroup}
 {\egroup
  \ifthenelse{\boolean{showsol}}
   {\vskip10pt\noindent\emph{Solution:}\vskip10pt \copy\answerbox}
   {\vskip \answerskip}}


\newcounter{question}
\setcounter{question}{0}

\newcommand{\question}{\addtocounter{question}{1}%
                          \pnum{\thequestion}\ }

\newcommand{\pnum}[1]{\noindent {\hskip -18pt \bfseries #1.}}

\makeatletter
\def\ps@myhead{%
    \let\@oddfoot\@empty\let\@evenfoot\@empty
    \def\@evenhead{\hfil{\large\leftmark}\hfil}%
    \def\@oddhead{\hfil{\large\rightmark}\hfil}%
    \let\@mkboth\@gobbletwo
    \let\sectionmark\@gobble
    \let\subsectionmark\@gobble
    \def\@oddfoot{\reset@font\hfil\thepage\hfil}
    \let\@evenfoot\@oddfoot
    }
\makeatother


\begin{document}
\thispagestyle{myhead}

\ifthenelse{\boolean{showsol}}
{
\markright{\bfseries Math B17 
- Winter 1999 - \ts{} (solutions)}
}
{
\markright{\bfseries B17 
- Winter 1999 - \ts{} }
}



%begin header
\ifthenelse{\boolean{showsol}}
{
\centerline{\Large SOLUTIONS}
}
{
\noindent \begin{minipage}[t]{3.5 in}
\Large
\begin{tabular}{|r|c|}\hline 
{\bfseries  Last Name:}&{ \ \ \ \ \ \ \ \ \ \ \ \ \ \ \ \ \ \ \ \ \ \ \ \ \ \ \ \ \ \ \ \ \ \ \ \ \ \ \ \ \ \ \ \ \ \ \ \ \ \ }\\ 
\hline 
{\bfseries First Name:}&{ \ \ \ \ \ \ \ \ \ \ \ \ \ \ \ \ \ \ \ \ \ \ \ \ \ \ \ \ \ \ \ \ \ \ \ \ \ \ \ \ \ \ \ \ \ \ \ \ \ \ }\\ 
\hline
\end{tabular}
\end{minipage}

\bigskip

}%end header
%
\bigskip


%%%%%%%%%%%%%%%%%%%%%%%%%%%%%%%%%%%%%%%%%%%%%%%%%%%%%%%%%%%%%%%%

\question 
Determine if the following infinite series converge 
or diverge:

\begin{enumerate}

\item[(a)]
$\displaystyle{\sum_{n=1000}^\infty\,\frac{(-1)^n}{\ln\ln\ln n}}$


\begin{answer}{2.5in}
It \emph{converges}, because it is an alternating series and the
$n$-th term is decreasing and tends to zero.
\bigskip
\end{answer}

\item[(b)]
$\displaystyle{\sum_{n=1}^\infty\,n\,\sin{\frac{1}{n}}}$

\begin{answer}{3in}
We have
\[
\lim_{n\to\infty}\,n\,\sin{\frac{1}{n}} =
\lim_{x\to 0}\,\frac{\sin x}{x} = 
\lim_{x\to 0}\,\frac{\cos x}{1} =
1
\]
so the series \emph{diverges} because the $n$-th term does not tend to
zero.
\end{answer}

\end{enumerate}

\clearpage

%%%%%%%%%%%%%%%%%%%%%%%%%%%%%%%%%%%%%%%%%%%%%%%%%%%%%%%%%%%%%%%%

\question 
Determine if the following infinite series converge or
diverge:

\begin{enumerate}

\item[(a)]
$\displaystyle{\sum_{n=1}^\infty\,\frac{1}{\sqrt{n^3+1}}}$

\begin{answer}{3in}
By comparison:
\[
\sum_{n=1}^\infty\,\frac{1}{\sqrt{n^3+1}} \leq
             \sum_{n=1}^\infty\,\frac{1}{n^{3/2}}
\]


The series on the right hand side converges by the integral test,
hence the given series \emph{converges}.
\bigskip
\end{answer}

\item[(b)]
$\displaystyle{\sum_{n=0}^\infty\,\frac{n^2+n+2}{n^3+3n+7}}$

\begin{answer}{3in}
We have 
\[
\lim_{n\to\infty}\,\frac{\frac{n^2+n+2}{n^3+3n+7}}{1/n}
=
\lim_{n\to\infty}\,\frac{n^3+n^2+2n}{n^3+3n+7} = 1
\]
so by the Limit Comparison Test the given series behaves the same as
$\displaystyle{\sum_{n=1}^\infty\,\frac{1}{n}},$ i.e.:
\emph{diverges}.

\bigskip
\end{answer}

\end{enumerate}

\clearpage

%%%%%%%%%%%%%%%%%%%%%%%%%%%%%%%%%%%%%%%%%%%%%%%%%%%%%%%%%%%%%%%%

\question 
Use power series to compute the following limit:

$\displaystyle{
\lim _{x\rightarrow 0}\left(\frac{1}{\ln (1+x)}-\frac{1}{x}\right)=}$


\begin{answer}{3in}
\[
\begin{aligned}
\lim _{x\to 0}\left(\frac{1}{\ln (1+x)}-\frac{1}{x}\right)
&=\lim _{x\to 0}\frac {x-\ln (1+x)}{x\,\ln (1+x)}\\
\noalign{\medskip}
&=\lim _{x\to 0}
\frac{x - \left(x-\frac{x^2}{2}+\frac{x^3}{3}-\cdots\right)}
     {x \, \left(x-\frac{x^2}{2}+\frac{x^3}{3}-\cdots\right)}\\
\noalign{\medskip}
&=\lim _{x\to 0}
\frac{\frac{x^2}{2}-\frac{x^3}{3}+\cdots}
     {x^2-\frac{x^3}{2}+\frac{x^4}{3}-\cdots}\\
\noalign{\medskip}
&=\lim _{x\to 0}
\frac{\frac{1}{2}-\frac{x}{3}+\cdots}
     {1-\frac{x}{2}+\frac{x^2}{3}-\cdots}\\
\noalign{\medskip}
&=\frac{1}{2}
\end{aligned}
\]
\end{answer}

\clearpage

%%%%%%%%%%%%%%%%%%%%%%%%%%%%%%%%%%%%%%%%%%%%%%%%%%%%%%%%%%%%%%%%

\question 
Solve the following system of equations or show that it has no
solutions:
\[
\left \{\begin {array}{rrrrrrrrr} 
x_{{1}}&+&2\,x_{{2}}&+&3\,x_{{3}}&+&4\,x_{{4}}&=& 10
\\\noalign{\medskip}
2\,x_{{1}}&+&3\,x_{{2}}&+&4\,x_{{3}}&+&5\,x_{{4}}&=& 14
\\\noalign{\medskip}
3\,x_{{1}}&+&4\,x_{{2}}&+&5\,x_{{3}}&+&6\,x_{{4}}&=& 18
\\\noalign{\medskip}
4\,x_{{1}}&+&5\,x_{{2}}&+&6\,x_{{3}}&+&7\,x_{{4}}&=& 22
\end {array}\right .
\]


\begin{answer}{3in}
The augmented matrix is
$\displaystyle{A' = 
\left [\begin {array}{rrrr} 1&2&3&4\\\noalign{\medskip}2&3&4&5
\\\noalign{\medskip}3&4&5&6\\\noalign{\medskip}4&5&6&7\end {array}
\right .
\hskip -2pt
\left|
\begin{array}{r}
10\\\noalign{\medskip}14\\\noalign{\medskip}18\\\noalign{\medskip}22
\end{array}
\right ].
}$

After using Gauss-Jordan reduction we get:
$\displaystyle{
\left [\begin {array}{rrrr} 1&0&-1&-2\\\noalign{\medskip}0&1&2&3
\\\noalign{\medskip}0&0&0&0\\\noalign{\medskip}0&0&0&0\end {array}
\right .
\hskip -2pt
\left|
\begin{array}{r}
-2\\\noalign{\medskip}6\\\noalign{\medskip}0\\\noalign{\medskip}0
\end{array}
\right ].
}$


i.e.:
\[
\left \{\begin {array}{rrrrrrrrr} 
x_{{1}}&&&-&x_{{3}}&-&2\,x_{{4}}&=&-2
\\\noalign{\medskip}
&&x_{{2}}&+&2\,x_{{3}}&+&3\,x_{{4}}&=&6
\end {array}\right .
\]

The solution is $x_1 = -2+x_3+2\,x_4$, $x_2=6-2\,x_3-3\,x_4$, or:
\[
\left [\begin {array}{r} x_{{1}}\\\noalign{\medskip}x_{{2}}
\\\noalign{\medskip}x_{{3}}\\\noalign{\medskip}x_{{4}}\end {array}
\right ]=\left [\begin {array}{r} -2\\\noalign{\medskip}6
\\\noalign{\medskip}0\\\noalign{\medskip}0\end {array}\right ]+x_{{3}}
\left [\begin {array}{r} 1\\\noalign{\medskip}-2\\\noalign{\medskip}1
\\\noalign{\medskip}0\end {array}\right ]+x_{{4}}\left [
\begin {array}{r} 2\\\noalign{\medskip}-3\\\noalign{\medskip}0
\\\noalign{\medskip}1\end {array}\right ]
\]

\end{answer}

\clearpage

%%%%%%%%%%%%%%%%%%%%%%%%%%%%%%%%%%%%%%%%%%%%%%%%%%%%%%%%%%%%%%%%

\question
Solve the following system of equations or show that it has no
solutions:
\[
\left \{\begin {array}{rrrrrrr} 
x_{{1}}&+&x_{{2}}&+&x_{{3}}&=& 1
\\\noalign{\medskip}
x_{{1}}&+&2\,x_{{2}}&+&3\,x_{{3}}&=& 4
\\\noalign{\medskip}
2\,x_{{1}}&+&3\,x_{{2}}&+&4\,x_{{3}}&=& 5
\\\noalign{\medskip}
&&x_{{2}}&+&2\,x_{{3}}&=& 0
\end {array}\right .
\]

\begin{answer}{3in}
The augmented matrix is:
$\displaystyle{
A' =
\left [\begin {array}{rrr} 1&1&1\\\noalign{\medskip}1&2&3
\\\noalign{\medskip}2&3&4\\\noalign{\medskip}0&1&2\end {array}
\right .
\hskip -2pt
\left|
\begin{array}{r}
1\\\noalign{\medskip}4\\\noalign{\medskip}5\\\noalign{\medskip}0
\end{array}
\right ].
}$

After applying Gauss-Jordan we get:
$\displaystyle{
\left [\begin {array}{rrr} 1&0&-1\\\noalign{\medskip}0&1&2
\\\noalign{\medskip}0&0&0\\\noalign{\medskip}0&0&0\end {array}
\right .
\hskip -2pt
\left|
\begin{array}{r}
0\\\noalign{\medskip}0\\\noalign{\medskip}1\\\noalign{\medskip}0
\end{array}
\right ].
}$

We see that the rank of the coefficient matrix is $2$, and the rank of
the augmented matrix is $3$, hence the system has no solutions.

\end{answer}

\clearpage

%%%%%%%%%%%%%%%%%%%%%%%%%%%%%%%%%%%%%%%%%%%%%%%%%%%%%%%%%%%%%%%%


\question Which of the following matrices are orthogonal? Why?
\[
A =
\left [\begin {array}{ccc} 1&1&0\\\noalign{\medskip}1&0&1
\\\noalign{\medskip}0&1&1\end {array}\right ]
\qquad
B =
\left [\begin {array}{rrr} 1&1&2\\\noalign{\medskip}1&-2&0
\\\noalign{\medskip}1&1&-2\end {array}\right ]
\qquad
C =
\left [\begin {array}{rrc} 
1/\sqrt{3}&1/\sqrt{6}&\phantom{-}1/\sqrt{2}\\
\noalign{\medskip}
1/\sqrt{3}&-2/\sqrt{6}&\phantom{-}0
\\\noalign{\medskip}
1/\sqrt{3}&1/\sqrt{6}&-1/\sqrt{2}
\end {array}\right ]
\]

\begin{answer}{3in}
A way to see if a matrix is orthogonal is to check if 
its inverse equals its transpose:
\[
A^t\,A =
\left [\begin {array}{ccc} 2&1&1\\\noalign{\medskip}1&2&1
\\\noalign{\medskip}1&1&2\end {array}\right ]
\]

Since $A^t\,A \neq I$, $A$ is not orthogonal.

\[
B^t\,B = 
\left [\begin {array}{ccc} 3&0&0\\\noalign{\medskip}0&6&0
\\\noalign{\medskip}0&0&8\end {array}\right ]
\]

So $B$ is not orthogonal either.

\[
C^t\,C = 
\left [\begin {array}{ccc} 1&0&0\\\noalign{\medskip}0&1&0
\\\noalign{\medskip}0&0&1\end {array}\right ]
\]

Hence $C$ is orthogonal.

\end{answer}


\clearpage


%%%%%%%%%%%%%%%%%%%%%%%%%%%%%%%%%%%%%%%%%%%%%%%%%%%%%%%%%%%%%%%%

\question In $\mathbb{R}^2$ find the change of basis matrix for a 
$60^\mathrm{0}$ clockwise rotation. What are the new coordinates of 
the point $(1,1)$? (Note: $\sin{\frac{\pi}{3}} = \frac{\sqrt{3}}{2}$, 
$\cos{\frac{\pi}{3}} = \frac{1}{2}$.)

\begin{answer}{3in}
The rotation transforms the standard basis 
$\displaystyle{
\left\{
\mathbf{e_1} = 
\left [\begin {array}{c} 1\\\noalign{\medskip}0\end {array}\right ],\,
\mathbf{e_2} = 
\left [\begin {array}{c} 0\\\noalign{\medskip}1\end {array}\right ]
\right\}
}$
into
$\displaystyle{
\left\{
\mathbf{v_1} = 
\left [\begin {array}{c} 
\phantom{-}{1}/{2}\\
\noalign{\medskip}
-{\sqrt{3}}/{2}\end {array}\right ],\,
\mathbf{v_2} = 
\left [\begin {array}{c} 
{\sqrt{3}}/{2}\\
\noalign{\medskip}{1}/{2}
\end {array}\right ]
\right\}, 
}$
hence the change of basis matrix is
\[
P = \left[\, \mathbf{v_1}\ \mathbf{v_2} \,\right]
=
\left [\begin {array}{cc} 
\phantom{-}1/2&\sqrt {3}/2\\
\noalign{\medskip}-\sqrt {3}/2&1/2
\end {array}\right ].
\]

Since $P$ is orthogonal, its inverse is its transpose:
$\displaystyle{
P^{-1} = P^{t} = 
\left [\begin {array}{cc} 
1/2&-\sqrt {3}/2\\
\noalign{\medskip}\sqrt {3}/2&\phantom{-}1/2
\end {array}\right ].
}$

The new coordinates of the point $(1,1)$ are:
\[
P^{-1}
\left [\begin {array}{c} 1\\\noalign{\medskip}1\end {array}\right ]
=
\left [\begin {array}{cc} 
1/2&-\sqrt {3}/2\\
\noalign{\medskip}\sqrt {3}/2&\phantom{-}1/2
\end {array}\right ]
\left [\begin {array}{c} 1\\\noalign{\medskip}1\end {array}\right ]
=
\left [\begin {array}{c} 
({1-\sqrt{3}})/{2}\\
\noalign{\medskip}
({1+\sqrt{3}})/{2}\\
\end {array}\right ],
\]
i.e.: 
$\displaystyle{
\left(\frac{1-\sqrt{3}}{2},\frac{1+\sqrt{3}}{2}\right)
}$.


\end{answer}

\clearpage


%%%%%%%%%%%%%%%%%%%%%%%%%%%%%%%%%%%%%%%%%%%%%%%%%%%%%%%%%%%%%%%%

\question Find the dimension and an \emph{orthonormal} basis for 
the column space of the following matrix:
\[A =
\left [\begin {array}{rrrr} 0&1&0&1\\\noalign{\medskip}1&0&-1&0
\\\noalign{\medskip}0&1&1&2\\\noalign{\medskip}1&0&1&2\end {array}
\right ]
\]

\solution{3in}{
After using Gauss reduction on $A$ the matrix becomes:
$\displaystyle{
\left [\begin {array}{rrrr} 1&0&-1&0\\\noalign{\medskip}0&1&0&1
\\\noalign{\medskip}0&0&1&1\\\noalign{\medskip}0&0&0&0\end {array}
\right ],
}$
which shows that the first three columns of $A$ form a basis 
for its column space:
\[
\left\{
\mathbf{v_1} = 
\left [\begin {array}{r} 0\\\noalign{\medskip}1\\\noalign{\medskip}0
\\\noalign{\medskip}1\end {array}\right ],
\mathbf{v_2} = 
\left [\begin {array}{r} 1\\\noalign{\medskip}0\\\noalign{\medskip}1
\\\noalign{\medskip}0\end {array}\right ],
\mathbf{v_3} = 
\left [\begin {array}{r} 0\\\noalign{\medskip}-1\\\noalign{\medskip}1
\\\noalign{\medskip}1\end {array}\right ]
\right\}
\]

Since the basis has three vectors, the dimension of the column space
is $3$.

Now we orthonormalize it by using the Gram-Schmidt process.
Since $\mathbf{v_1}$ is already orthogonal to $\mathbf{v_2}$ 
and $\mathbf{v_3}$, we need to apply Gram-Schmidt to 
$\{\mathbf{v_2},\mathbf{v_3}\}$ only: 
\[
\mathbf{v'_3} = \mathbf{v_3} - 
\frac{\mathbf{v_2}\cdot\mathbf{v_3}}{\mathbf{v_2}\cdot\mathbf{v_2}}
\,\mathbf{v_2}
=
\left [\begin {array}{r} 0\\\noalign{\medskip}-1\\\noalign{\medskip}1
\\\noalign{\medskip}1\end {array}\right ]
-
\frac{1}{2}
\left [\begin {array}{r} 1\\\noalign{\medskip}0\\\noalign{\medskip}1
\\\noalign{\medskip}0\end {array}\right ]
=
\left [\begin {array}{c} -1/2\\\noalign{\medskip}-1
\\\noalign{\medskip}\phantom{-}1/2
\\\noalign{\medskip}\phantom{-}1\end {array}\right ].
\]

After normalizing we get the following orthonormal basis:
\[
\left\{
\mathbf{u_1} =
\frac{1}{\sqrt{2}} 
\left [\begin {array}{r} 0\\\noalign{\medskip}1\\\noalign{\medskip}0
\\\noalign{\medskip}1\end {array}\right ],
\mathbf{u_2} =
\frac{1}{\sqrt{2}} 
\left [\begin {array}{r} 1\\\noalign{\medskip}0\\\noalign{\medskip}1
\\\noalign{\medskip}0\end {array}\right ],
\mathbf{u_3} = 
\frac{1}{\sqrt{10}} 
\left [\begin {array}{c} -1\\\noalign{\medskip}-2
\\\noalign{\medskip}\phantom{-}1
\\\noalign{\medskip}\phantom{-}2\end {array}\right ]
\right\}.
\]


}


\clearpage

%%%%%%%%%%%%%%%%%%%%%%%%%%%%%%%%%%%%%%%%%%%%%%%%%%%%%%%%%%%%%%%%

\question Find the principal axes and classify the central conic:
\[
5\,x^2 + 5\,y^2 - 6\,xy = 8
\]

\solution{3in}{
The conic can be represented as 
$\displaystyle{
\left [\begin{array}{cc} x&y\end{array}\right ]
A
\left [\begin {array}{c} x\\\noalign{\medskip}y\end {array}\right ]
= 8},$
where 
$\displaystyle{
A = 
\left [\begin {array}{rr} 5&-3\\\noalign{\medskip}-3&5\end {array}
\right ] .
}$

We must diagonalize $A$ as $D = P^{t} A P$ for some \emph{orthogonal} 
matrix 
$\displaystyle{P = 
\left [\begin{array}{cc} \mathbf{u_1}&\mathbf{u_2}\end{array}\right ]}
$,
where $\{\mathbf{u_1},\mathbf{u_2}\}$ is an orthonormal basis for 
$\mathbb{R}^2$ consisting of eigenvectors for $A$. 

The eigenvalues of $A$ are the roots of the characteristic polynomial:
\[
\det\left(A-\lambda\,I\right) =
\det 
\left [\begin {array}{cc} 5-\lambda&-3\\
\noalign{\medskip}-3&5-\lambda\end {array}
\right ]
= {\lambda}^{2}-10\,\lambda+16
= (\lambda -2)\,(\lambda-8)
\]

The eigenvalues are $\lambda=2$ and $\lambda=8$.

For $\lambda=2$ we must solve 
$\displaystyle{
\left [\begin {array}{rr} 3&-3\\
\noalign{\medskip}-3&3\end {array}
\right ]
\left [\begin {array}{c} x_1\\\noalign{\medskip}x_2
\end {array}\right ]
=
\left [\begin {array}{c} 0\\\noalign{\medskip}0
\end {array}\right ]
}$.
The solution is $x_1 = x_2$, or:
$\displaystyle{
\left [\begin {array}{r} x_{{1}}\\\noalign{\medskip}x_{{2}}
\end {array}\right ]=x_{{2}}\left [\begin {array}{c} 1
\\\noalign{\medskip}1\end {array}\right ]
}$, so we take 
$\displaystyle{\mathbf{v_1} = 
\left [\begin {array}{c} 1
\\\noalign{\medskip}1\end {array}\right ]
}$ as eigenvector.

For $\lambda=8$ we must solve 
$\displaystyle{
\left [\begin {array}{rr} -3&-3\\
\noalign{\medskip}-3&-3\end {array}
\right ]
\left [\begin {array}{c} x_1\\\noalign{\medskip}x_2
\end {array}\right ]
=
\left [\begin {array}{c} 0\\\noalign{\medskip}0
\end {array}\right ]
}$.
The solution is $x_1 = -x_2$, or:
$\displaystyle{
\left [\begin {array}{c} x_{{1}}\\\noalign{\medskip}x_{{2}}
\end {array}\right ]=x_{{2}}\left [\begin {array}{c} -1
\\\noalign{\medskip}1\end {array}\right ]
}$, so we take 
$\displaystyle{\mathbf{v_2} = 
\left [\begin {array}{r} -1
\\\noalign{\medskip}1\end {array}\right ]
}$.

Note that $\mathbf{v_1}$ and $\mathbf{v_2}$ are already orthogonal, 
so all we need is to normalize them:
$\mathbf{u_1} = \frac{1}{\sqrt{2}}\,\mathbf{v_1}$,
$\mathbf{u_2} = \frac{1}{\sqrt{2}}\,\mathbf{v_2}$.
The matrix for the change of basis is:
\[
P = 
\left[ \begin{array}{cc} \mathbf{u_1} & \mathbf{u_2} \end{array} \right] 
= 
\frac{1}{\sqrt{2}}
\left [\begin {array}{rr} 1&-1\\\noalign{\medskip}1&1\end {array}
\right ].
\]

In the new basis the conic is 
$\displaystyle{
\left [\begin{array}{cc} x'&y'\end{array}\right ]
D
\left [\begin {array}{c} x'\\\noalign{\medskip}y'\end {array}\right ]
= 8},$
where 
\[
D = P^{t}\,A\,P =
\left [\begin {array}{cc} 2&0\\\noalign{\medskip}0&8\end {array}
\right ],
\]
and
\[
\left [\begin {array}{c} x'\\\noalign{\medskip}y'\end {array}\right ]
=
P^{t} 
\left [\begin {array}{c} x\\\noalign{\medskip}y\end {array}\right ]
=
\frac{1}{\sqrt{2}}
\left [\begin {array}{rr} 1&1\\
\noalign{\medskip}-1&1\end {array}\right ]
\left [\begin {array}{c} x\\\noalign{\medskip}y\end {array}\right ]
\]
i.e.:
\[
\left\{
\begin{aligned}
x' &= \tfrac{1}{\sqrt{2}}\,\left(x+y\right)\\
y' &= \tfrac{1}{\sqrt{2}}\,\left(-x+y\right)
\end{aligned}
\right.
\]

Hence the conic is $2\,{x'}^2+8\,{y'}^2 = 8$, or equivalently:
$\displaystyle{
\frac{{x'}^2}{4} + {y'}^2 = 1
}$, which is an \emph{ellipse}.
Its principal axes are given by the basic vectors
\[
\mathbf{u_1} = \frac{1}{\sqrt{2}}
\left [\begin {array}{r} 1
\\\noalign{\medskip}1\end {array}\right ] , \qquad
\mathbf{u_2} = \frac{1}{\sqrt{2}}
\left [\begin {array}{r} -1
\\\noalign{\medskip}1\end {array}\right ] .
\]

Note: An alternative solution is 
$\displaystyle{
{x'}^2 + \frac{{y'}^2}{4}  = 1
}$, and
\[
\mathbf{u_1} = \frac{1}{\sqrt{2}}
\left [\begin {array}{r} -1
\\\noalign{\medskip}1\end {array}\right ] , \qquad
\mathbf{u_2} = \frac{1}{\sqrt{2}}
\left [\begin {array}{r} 1
\\\noalign{\medskip}1\end {array}\right ] .
\]


}

\clearpage

%%%%%%%%%%%%%%%%%%%%%%%%%%%%%%%%%%%%%%%%%%%%%%%%%%%%%%%%%%%%%%%%

\question Let 
$\displaystyle{ A=\left [\begin {array}{ccc}
1&2&-2\\\noalign{\medskip}2&1&-2 \\\noalign{\medskip}0&0&-1
\end{array}\right ].}$ 
Find a matrix $P$ such that $D = P^{-1}\,A\,P$ 
is diagonal.

\solution{3in}{ 
The solution is of the form $P = [\,\mathbf{v_1\ v_2\ v_3}\,]$, 
were $\{\mathbf{v_1},\mathbf{v_2},\mathbf{v_3}\}$ is a 
basis for $\mathbb{R}^3$ consisting of eigenvectors for $A$.

We have
\[
\det(A-\lambda\,I) = 
\det\left [\begin {array}{ccc} 1-\lambda&2&-2\\\noalign{\medskip}2&1-
\lambda&-2\\\noalign{\medskip}0&0&-1-\lambda\end {array}\right ] =
3 + 5\lambda +\lambda^2 - \lambda^3
= - (\lambda - 3)(\lambda + 1)^2
\]

So the roots of the characteristic polynomial are $\lambda=3$ and
$\lambda=-1$ (double).

For $\lambda=3$ we get:
$\displaystyle{A - 3\,I =
\left [\begin {array}{rrr} -2&2&-2\\\noalign{\medskip}2&-2&-2
\\\noalign{\medskip}0&0&-4\end {array}\right ]
}$

After using Gauss-Jordan that matrix becomes: 
$\displaystyle{
\left [\begin {array}{rrr} 1&-1&0\\\noalign{\medskip}0&0&1
\\\noalign{\medskip}0&0&0\end {array}\right ]
}$

The solution of $(A-3\,I)\,\mathbf{x} = \mathbf{0}$ is
$x_1=x_2$, $x_3=0$, i.e.:
\[
\left [\begin {array}{r} x_{{1}}\\\noalign{\medskip}x_{{2}}
\\\noalign{\medskip}x_{{3}}\end {array}\right ]=x_{{2}}\left [
\begin {array}{r} 1\\\noalign{\medskip}1\\\noalign{\medskip}0
\end {array}\right ]
\]

So we take 
$\displaystyle{\mathbf{v_1} = 
\left [
\begin {array}{r} 1\\\noalign{\medskip}1\\\noalign{\medskip}0
\end {array}\right ]}$
as the first eigenvector.

Next, for $\lambda = -1$ we get
$\displaystyle{
A + I = 
\left [\begin {array}{rrr} 2&2&-2\\\noalign{\medskip}2&2&-2
\\\noalign{\medskip}0&0&0\end {array}\right ].
}$

After using Gauss-Jordan that matrix becomes 
$\displaystyle{
\left [\begin {array}{rrr} 1&1&-1\\\noalign{\medskip}0&0&0
\\\noalign{\medskip}0&0&0\end {array}\right ].
}$

Hence, the solution of $(A+I)\,\mathbf{x} = \mathbf{0}$ is 
$x_1 = -x_2+x_3$, i.e.:
\[
\left [\begin {array}{r} x_{{1}}\\\noalign{\medskip}x_{{2}}
\\\noalign{\medskip}x_{{3}}\end {array}\right ]=x_{{2}}\left [
\begin {array}{r} -1\\\noalign{\medskip}1\\\noalign{\medskip}0
\end {array}\right ]+x_{{3}}\left [\begin {array}{r} 1
\\\noalign{\medskip}0\\\noalign{\medskip}1\end {array}\right ]
\]

So, we take the eigenvectors 
$\displaystyle{\mathbf{v_2} = 
\left [
\begin {array}{r} -1\\\noalign{\medskip}1\\\noalign{\medskip}0
\end {array}\right ]}$
and
$\displaystyle{\mathbf{v_3} = 
\left [
\begin {array}{r} 1\\\noalign{\medskip}0\\\noalign{\medskip}1
\end {array}\right ]}$.

The matrix $P$ is:\footnote{
Other solutions, obtained by permuting the columns of 
$P$, are also possible. Also the vectors $\mathbf{v_2}$ 
and $\mathbf{v_2}$ could be chosen differently, provided 
they span the same subspace.}

$\displaystyle{P = [\,\mathbf{v_1\ v_2\ v_3}\,] =
\left [\begin {array}{rrr} 1&-1&1\\\noalign{\medskip}1&1&0
\\\noalign{\medskip}0&0&1\end {array}\right ],
}$

and
\[
D = P^{-1}\,A\,P = 
\left [\begin {array}{ccc} 3&0&0\\\noalign{\medskip}0&-1&0
\\\noalign{\medskip}0&0&-1\end {array}\right ]
\]


}

\clearpage

%%%%%%%%%%%%%%%%%%%%%%%%%%%%%%%%%%%%%%%%%%%%%%%%%%%%%%%%%%%%%%%%

\end{document}







